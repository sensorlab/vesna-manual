\documentclass[a4paper, 10pt]{article}

% \usepackage{verbatim}

\usepackage[utf8]{inputenc}

\usepackage{graphicx}

\usepackage{float}

\usepackage{hyperref}

\title{SensorLab VESNA open source development environment setup manual for Linux based development}
\author{Zoltan Padrah}

\begin{document}

% title page
\begin{titlepage}
    \begin{center}
    \textbf{
        \raisebox{-0.4\height}{
            \includegraphics[width=1cm]{./install-guide-linux-images/jsi-logo.png}
        }
        \Large ``Jožef Stefan'' Institute \\[2mm]
        \raisebox{-0.4\height}{
            \includegraphics[width=1cm]{./install-guide-linux-images/comms-logo.png}
        }
        Department of Communication Systems\\[2mm]
        \includegraphics[height=1cm]{./install-guide-linux-images/sensorlab-logo.png}
        }\\

    \vfill

    \textbf{\huge SensorLab VESNA open source development environment setup manual
        for Linux based development}\\[1.5cm]


    \textbf{ \Large
    version 0.11}\\[1cm]

    \textnormal{\Large
    by Zoltan Padrah, Tomaž Šolc}\\[1cm]

    \vfill

    \textnormal{\large
    \today\\[1cm]
    }
    \end{center}
\end{titlepage}


\section{Introduction}

This document describes the installation of a cross-compiler and associated
development tools required to compile and debug software for VESNA wireless
sensor node on a GNU/Linux operating system.

It offers two distinct approaches to development: one that focuses on graphical
interaction and an integrated development environment (IDE) and one focused on
command-line interfaces and scripting. A short overview of both is given below
- they are identical in regard to functionality and it is up to the reader to
choose the approach that he or she finds most convenient.

The graphical approach is recommended for new developers and those moving from a
Windows development environment. Developers with a background in free open
source software development will likely prefer the command-line way.

The exact steps are performed on a specific distribution mentioned, but the
document should be applicable to any distribution. Besides distribution-specific
instructions, generic steps will be also described.

\subsection{Graphical way}

\begin{enumerate}
\item Eclipse IDE with EGit, Zylin Embedded CDT
\item Cross-compiler provided by CodeSourcery Lite binary toolchain
	distribution
\item OpenOCD on-chip debugger
\item moserial or CuteCom serial terminal
\item Steps performed on Ubuntu 10.10, 32 bit distribution
\end{enumerate}

\subsection{Command-line way}

\begin{enumerate}
\item Text editor and terminal emulator of choice, GNU make, GNU debugger
\item Cross-compiler compiled from official GNU sources
\item OpenOCD on-chip debugger
\item Steps performed on Debian 6.0 (Squeeze), amd64 architecture
\end{enumerate}

\newpage

\tableofcontents
\newpage

\listoffigures
\newpage

\subsection*{List of Abbreviations}
	\begin{tabular}{ l l }
    CDT	    & C/C++ Development Tooling \\
%    SVN     & Apache Subversion \\
	SSH		& Secure Shell \\
	\end{tabular}
\newpage


\section{Graphical way}

\subsection{Preparing the system and installing custom software}

% This section includes special steps, that

\subsubsection{Setting up bash as the default shell in the system}

In order to be able to run in Eclipse external programs from special locations
(in our case, the compiler and other needed tools),
the default shell for script execution must be changed.
Otherwise the default system shell (dash) will be launched by Eclipse,
and the compiler won't be found.

This can be achieved by launching Applications $\rightarrow$ Accessories $\rightarrow$ Terminal,
copying the following command in it:

\begin{verbatim}
sudo dpkg-reconfigure dash
\end{verbatim}

and pressing Enter.

You will be asked for your password,
and after that a question will be presented, asking
if dash should be installed as default system shell.
See Figure \ref{fig:set-dash-default}.
Select \emph{No}, and press enter.

    \begin{figure}[H]
    \centering
        \includegraphics[width=0.9\textwidth]{./install-guide-linux-images/set-dash-default.png}
        \caption{Setting the default shell to bash, instead of dash}
        \label{fig:set-dash-default}
    \end{figure}

Finally close the terminal window.
In order to apply these changes, you must either log out and log in,
or launch Eclipse from a terminal.
Otherwise you will get errors when you are trying to compile the source code.
It is recommended to log out and log in after
competing section \ref{sect:change-path}, in order to avoid one extra logout and login.

\subsubsection{Installing the CodeSourcery Lite toolchain}

The CodeSourcery Lite toolchain contains
the compiler, linker, debugger, and other utilities needed for
ARM development.

Go to
\url{http://www.codesourcery.com/sgpp/lite/arm/portal/subscription?@template=lite},
select \emph{EABI} as Target OS, and click on the
\emph{All versions} link.
% \emph{Sourcery G++ Lite \textit{version}} link.
As of date of writing, the latest version is 2010.09-51, but
due to compatibility problems during debugging, the recommended version is \emph{2010q1-188}.
For details, see section \ref{sect:gdb-problems} in Troubleshooting part.
The direct link to the recommended version is
\url{https://sourcery.mentor.com/GNUToolchain/package6493/public/arm-none-eabi/arm-2010q1-188-arm-none-eabi-i686-pc-linux-gnu.tar.bz2}.

On the webpage, select \emph{IA32 GNU/Linux TAR},
and save the received file.
Next, open the file, and extract its contents into your home folder.
This way, you should have a folder named arm-\textit{version} in your home directory.
In this case, the folder is called arm-2010q1.

The next step is to make the system know about the location of the tools;
this is achieved by setting up the PATH for the current user.

\subsubsection{Inserting the directory of the CodeSourcery toolchain in the PATH}

    \label{sect:change-path}
The \verb+PATH+ variable stores a list of directories, where various
program executables can be found. In order to make Eclipse know about the
ARM development tools, the directory containing the tools must be inserted
into this list.

Select from the menu Applications $\rightarrow$ Accessories $\rightarrow$ Text Editor,
and click open. Navigate to your home folder, then right click on the file list,
and select from the menu \emph{Show hidden files}. This operation is depicted on Figure
\ref{fig:show-hidden-files}.

    \begin{figure}[H]
    \centering
        \includegraphics[width=\textwidth]{./install-guide-linux-images/show-all-files.png}
        \caption{Opening hidden files for editing}
        \label{fig:show-hidden-files}
    \end{figure}

From the list, select the file \verb+.bashrc+. Note that hidden files in Linux
have their name starting with a dot (.).

Select the file called \verb+.bashrc+ and insert the following text in it,
close to the beginning:

\begin{verbatim}
export PATH=$PATH:$HOME/arm-2010q1/bin
\end{verbatim}

See Figure \ref{fig:editing-bashrc} for example. If you selected different CodeSourcery version,
then replace the text \verb+2010q1+ accordingly.

Note: the new line in the file needs to be placed before the block
\emph{\# If not running interactively, don't do anything},
in order to change the PATH variable in all situations,
including the case when compiling programs from Eclipse.

    \begin{figure}[H]
    \centering
        \includegraphics[width=\textwidth]{./install-guide-linux-images/editing-bashrc.png}
        \caption{Editing .bashrc}
        \label{fig:editing-bashrc}
    \end{figure}

Save the file and close the text editor.
After competing these operations, and before launching Eclipse, it is needed to
log out and log in back, in order to apply the performed changes.

\subsection{Installing programs provided as packages by the distribution}

Programs provided by Linux distributions can be installed by using the package management system.
On Ubuntu, the Synaptic package manager is recommended.
Other distributions have equivalent tools for package management.
If Synaptic is not installed, the you can install it first by using
Ubuntu Software Center.

Open Synaptic by selecting in the menu System $\rightarrow$ Administration $\rightarrow$
Synaptic Package Manager,
or Dash home, and write Synaptic in the search field.

\subsubsection{Eclipse}

In Synaptic select Edit $\rightarrow$ Search, or Ctrl+F, and enter \verb+eclipse+
in the search box. Right click the package called
\verb+eclipse-cdt+ and select \emph{Mark for installation}.
See Figure \ref{fig:synaptic-eclipse}.

    \begin{figure}[H]
    \centering
        \includegraphics[width=\textwidth]{./install-guide-linux-images/synaptic-eclipse.png}
        \caption{Selecting eclipse-cdt for installation}
        \label{fig:synaptic-eclipse}
    \end{figure}

After this a dialog should pop up, asking if the other, needed packages should be installed,
see Figure \ref{fig:synaptic-eclipse-extra-packages.png}. Click \emph{Mark}.

    \begin{figure}[H]
    \centering
        \includegraphics[width=0.5\textwidth]{./install-guide-linux-images/synaptic-eclipse-extra-packages.png}
        \caption{Extra packages required by eclipse-cdt}
        \label{fig:synaptic-eclipse-extra-packages.png}
    \end{figure}

\subsubsection{Eclipse Git plugin (EGit)}

From the search results in synaptic, also select the package
\verb+eclipse-egit+ for installation, see
Figure \ref{fig:synapic-eclipse-egit.png}.
EGit is used in Eclipse for supporting the Git version control system.

    \begin{figure}[H]
    \centering
        \includegraphics[width=0.9\textwidth]{./install-guide-linux-images/synaptic-eclipse-egit.png}
        \caption{Selecting eclipse-egit for installation}
        \label{fig:synapic-eclipse-egit.png}
    \end{figure}

\subsubsection{OpenOCD}

Still in Synaptic, search for package called \verb+openocd+,
and mark it for installation, as shown on Figure \ref{fig:synaptic-openocd}.


    \begin{figure}[H]
    \centering
        \includegraphics[width=0.8\textwidth]{./install-guide-linux-images/synaptic-openocd.png}
        \caption{Selecting openocd for installation}
        \label{fig:synaptic-openocd}
    \end{figure}


\subsubsection{CuteCom}

Optionally, install a program for communicating on the RS232 port,
connected to VESNA.
CuteCom is one viable option, Moserial is another one.
You can use any of these programs; note that CuteCom handles long
RS232 captures much better than Moserial.

Search for \verb+cutecom+ on Synaptic, and select it for installation.
See Figure \ref{fig:synaptic-cutecom}.

    \begin{figure}[H]
    \centering
        \includegraphics[width=0.9\textwidth]{./install-guide-linux-images/synaptic-cutecom.png}
        \caption{Selecting cutecom for installation}
        \label{fig:synaptic-cutecom}
    \end{figure}

\subsubsection{Moserial}

Search for \verb+moserial+ on Synaptic, and select it for installation.
See Figure \ref{fig:synaptic-moserial}.


    \begin{figure}[H]
    \centering
        \includegraphics[width=0.9\textwidth]{./install-guide-linux-images/synaptic-moserial.png}
        \caption{Selecting moserial for installation}
        \label{fig:synaptic-moserial}
    \end{figure}


The next step is to actually perform the installation of the above mentioned programs.

\subsubsection{Git}

Optionally you can install the programs to work with Git version control system.
Inside Eclipse Git is handled by the EGit plugin, but sometimes it can be useful
to manipulate Git repositories from outside Eclipse.

To install git, search in synaptic for \verb+git gui+ and
install the \verb+git-gui+ package for installation.
This package will install as dependencies all of the commonly useful Git tools.

\subsubsection{Performing the actual installation}

Finally, click \emph{Apply} in Synaptic.
A window as the one shown on Figure \ref{fig:synaptic-apply-changes.png}
should appear.

    \begin{figure}[H]
    \centering
        \includegraphics[width=0.8\textwidth]{./install-guide-linux-images/synaptic-apply-changes.png}
        \caption{Selecting openocd for installation}
        \label{fig:synaptic-apply-changes.png}
    \end{figure}

Click \emph{Apply}, wait for the installation to finish, and close Synaptic.

\subsection{Installing Eclipse plugins}

First, start Eclipse, by selecting in the menu Applications $\rightarrow$
Programming $\rightarrow$ Eclipse,
or Dash home, and search for Eclipse.

After Eclipse starts, it will ask for a workspace, as shown on Figure \ref{fig:eclipse-workspace}.

    \begin{figure}[H]
    \centering
        \includegraphics[width=0.8\textwidth]{./install-guide-linux-images/eclipse-workspace.png}
        \caption{Selecting Eclipse workspace}
        \label{fig:eclipse-workspace}
    \end{figure}

The default workspace location should be good in most cases, so click \emph{OK}.

In Eclipse, select in the menu Help $\rightarrow$ Install new Software..., as shown in
Figure \ref{fig:eclipse-update}.

    \begin{figure}[H]
    \centering
        \includegraphics[width=0.9\textwidth]{./install-guide-linux-images/eclipse-update.png}
        \caption{Search for Eclipse plugins to install}
        \label{fig:eclipse-update}
    \end{figure}


% Galileo Update Site - http://download.eclipse.org/releases/galileo/

% NOTE: why don't use Eclipse Hardware debugger instead of Zylin CDT?
% http://www.eclipse.org/forums/index.php?t=msg&goto=536316&
% http://download.eclipse.org/tools/cdt/releases/galileo/
% https://sites.google.com/site/stm32discovery/open-source-development-with-the-stm32-discovery/getting-hardware-debuging-working-with-eclipse-and-code-sourcey
%
% @NOTE: I think at the time of creating the development environment,
% these tools were not available yet. For now it is not worth fixing something
% that is working, but at some point we should experiment with better tools.

\subsubsection{GNU ARM C/C++ development support}

Again, select in the menu Help $\rightarrow$ Install new Software... as shown in
Figure \ref{fig:eclipse-update}.

In the install window, paste in the \emph{Work with} field
the following URL:
\url{http://gnuarmeclipse.sourceforge.net/updates} and press Enter.
The list in the middle of the window should have an item, as shown on
Figure \ref{fig:eclipse-gnu-arm}.

    \begin{figure}[H]
    \centering
        \includegraphics[width=0.9\textwidth]{./install-guide-linux-images/eclipse-gnu-arm.png}
        \caption{Selecting GNU ARM C/C++ development for Eclipse}
        \label{fig:eclipse-gnu-arm}
    \end{figure}

Check the \emph{GNU ARM C/C++ Development Plugin} and click \emph{Next},
\emph{Next}, select that you accept the license and click \emph{Finish}.

Finally approve the installation of unsigned software,
and restart Eclipse when it's asking.

\subsubsection{Zylin Embedded CDT}

In Eclipse, select again in the menu Help $\rightarrow$ Install new Software...

In the \emph{Work with} field, enter the following URL:
\url{http://opensource.zylin.com/zylincdt}.

Select the \emph{Zylin Embedded CDT} plugin for installation,
as depicted on Figure \ref{fig:eclipse-zylin}.

    \begin{figure}[H]
    \centering
        \includegraphics[width=0.9\textwidth]{./install-guide-linux-images/eclipse-zylin.png}
        \caption{Selecting Zylin Embedded CDT plugin for Eclipse}
        \label{fig:eclipse-zylin}
    \end{figure}

Proceed with the installation as for GNU ARM C/C++ development support plugin.

\subsection{Setting up authentication for GitHub}

In order to download the source code from GitHub, one needs to
have a GitHub account and at least one SSH key pair set up for GitHub.

    \begin{figure}[H]
    \centering
        \includegraphics[width=0.8\textwidth]{./install-guide-linux-images/auth-copied.png}
        \caption{SSH keys at $ \sim/.ssh $ }
        \label{fig:auth-copied}
    \end{figure}

In case you have the SSH key pair set up, then you can copy the key to
the \verb+~/.ssh/+ folder,
add the keys to EGit's list of keys and you are done.
Note that you need to set up the correct permissions on the key files:
the files must be owned by your user, and they can be readable only to
your user.

If you need to create an account and generate keys, then you can use the
tool available in Eclipse, and proceed with account creation on GitHub.
The key generator tool can be found at
\emph{Window} $\rightarrow$ \emph{Preferences...}
and search for \emph{SSH2}, then select
\emph{General} $\rightarrow$ \emph{Network Connection} $\rightarrow$
\emph{SSH2} $\rightarrow$ \emph{Key Management}.

    \begin{figure}[H]
    \centering
        \includegraphics[width=0.8\textwidth]{./install-guide-linux-images/auth-generate.png}
        \caption{Generating SSH keys with Eclipse}
        \label{fig:auth-generate}
    \end{figure}

For adding private SSH keys to Eclipse, go to
\emph{Window} $\rightarrow$ \emph{Preferences...}
and search for \emph{SSH2}, then select
\emph{General} $\rightarrow$ \emph{Network Connection} $\rightarrow$
\emph{SSH2} $\rightarrow$ \emph{General}.
There click \emph{Add private key...} and select the private key that you
will use with GitHub.

    \begin{figure}[H]
    \centering
        \includegraphics[width=0.8\textwidth]{./install-guide-linux-images/auth-add-key.png}
        \caption{Adding private keys to Eclipse}
        \label{fig:auth-add-key}
    \end{figure}

\subsection{Testing the setup}


Now everything should be installed for software development.
This section guides through the testing of the installed programs
and gives a short introductions to their functioning.

Note that after installing the compiler tools, you will need to log out
and log in before launching Eclipse. Otherwise you might get errors
saying that the executable \verb+arm-none-eabi-gcc+ cannot be found.

\subsubsection{Creating a new project and importing the source code}

In Eclipse, select \emph{File} $\rightarrow$ \emph{Import...}.
For the import source select \emph{Projects from Git}, as shown on
Figure \ref{fig:test-prj-git}.

    \begin{figure}[H]
    \centering
        \includegraphics[width=0.8\textwidth]{./install-guide-linux-images/test-prj-git.png}
        \caption{Import from Git}
        \label{fig:test-prj-git}
    \end{figure}

Click \emph{Next}.

    \begin{figure}[H]
    \centering
        \includegraphics[width=0.9\textwidth]{./install-guide-linux-images/test-select-git-repo.png}
        \caption{Select the Git repository}
        \label{fig:test-select-git-repo}
    \end{figure}

Because the source code is not available in a local git repository, first
download the source code by cloning the
\verb+vesna-drivers+ repository.


Click \emph{Clone...} and fill in the URI field with
\verb+git@github.com:sensorlab/vesna-drivers.git+

    \begin{figure}[H]
    \centering
        \includegraphics[width=0.9\textwidth]{./install-guide-linux-images/test-cloning-settings.png}
        \caption{Parameters of the repository to clone}
        \label{fig:test-cloning-settings}
    \end{figure}

Click \emph{Next}.

    \begin{figure}[H]
    \centering
        \includegraphics[width=0.9\textwidth]{./install-guide-linux-images/test-select-branches.png}
        \caption{Selecting branches to clone}
        \label{fig:test-select-branches}
    \end{figure}

Click \emph{Next}.

    \begin{figure}[H]
    \centering
        \includegraphics[width=0.9\textwidth]{./install-guide-linux-images/test-clone-location.png}
        \caption{Selecting the location of local repository}
        \label{fig:test-clone-location}
    \end{figure}

For the destination directory select the
vesna-drivers directory inside eclipse workspace.
This way it's easier to manage the project.
Click \emph{Finish}

Now you can select the cloned repository, from which the project
can be imported.

    \begin{figure}[H]
    \centering
        \includegraphics[width=0.9\textwidth]{./install-guide-linux-images/test-select-repo.png}
        \caption{Selecting the repository from which to import a project}
        \label{fig:test-select-repo}
    \end{figure}

Click \emph{Next}.
Select the \emph{Use new projects wizard}.

    \begin{figure}[H]
    \centering
        \includegraphics[width=0.9\textwidth]{./install-guide-linux-images/test-select-wizard.png}
        \caption{Selecting the wizard that imports a project}
        \label{fig:test-select-wizard}
    \end{figure}

Click \emph{Finish}.
From the project types, select \emph{C/C++} $\rightarrow$
\emph{Makefile Project with Existing Code}.

    \begin{figure}[H]
    \centering
        \includegraphics[width=0.9\textwidth]{./install-guide-linux-images/test-select-project-type.png}
        \caption{Selecting the project type}
        \label{fig:test-select-project-type}
    \end{figure}

Clink \emph{Next}.
Set the name of the project to \verb+vesna-drivers+.

    \begin{figure}[H]
    \centering
        \includegraphics[width=0.9\textwidth]{./install-guide-linux-images/test-project-name.png}
        \caption{Selecting the project name}
        \label{fig:test-project-name}
    \end{figure}

Click \emph{Finish}.

    \begin{figure}[H]
    \centering
        \includegraphics[width=0.9\textwidth]{./install-guide-linux-images/test-got-project.png}
        \caption{Checking out the source as project}
        \label{fig:test-got-project}
    \end{figure}

Now you should have the sample project in Eclipse, see
Figure \ref{fig:test-got-project}.

\subsubsection{Compiling the source}

In Eclipse, open the C/C++ perspective,  by selecting \emph{Window} $\rightarrow$
\emph{Open perspective} $\rightarrow$ Other...,
and clicking \emph{C/C++} and \emph{OK}.
See Figure \ref{fig:compile-c-perspective}.

    \begin{figure}[H]
    \centering
        \includegraphics[width=0.8\textwidth]{./install-guide-linux-images/compile-c-perspective.png}
        \caption{Open the C/C++ perspective}
        \label{fig:compile-c-perspective}
    \end{figure}

In this perspective,
on the right side, there should be a \emph{Make targets} view.

    \begin{figure}[H]
    \centering
        \includegraphics[width=0.8\textwidth]{./install-guide-linux-images/compile-make.png}
        \caption{Make targets view}
        \label{fig:compile-make}
    \end{figure}

Select the \emph{VESNADriversDemo} folder in the Make targets view,
and right-click on it and select \emph{New} in the menu.
Set the make target name to \verb+all+.

    \begin{figure}[H]
    \centering
        \includegraphics[width=0.4\textwidth]{./install-guide-linux-images/compile-add-target.png}
        \caption{Adding a make target}
        \label{fig:compile-add-target}
    \end{figure}

Other useful targets are \verb+clean+ for removing the build output files from the source 
code and \verb+drivers_demo.load5+ for loading the application to VESNA.

Click \emph{OK}, then double-click the newly created make target.
If everything goes well, you should see similar lines in the console, as the text below:

    \begin{figure}[H]
    \centering
        \includegraphics[width=\textwidth]{./install-guide-linux-images/compile-result.png}
        \caption{Console window contents, after a successful compilation}
        \label{fig:compile-result}
    \end{figure}

Now the example project has compiled, and it's ready to be loaded on VESNA.

\subsubsection{Uploading to the device}


First, connect all the necessary cables to the debugger:
    \begin{itemize}
    \item RS232 cable
    \item JTAG cable
    \item power supply
    \end{itemize}
and plug the debugger into a USB slot on the PC.

Then, it's recommended to launch the RS232 monitoring program.
In order to do this, select from the menu
% Applications $\rightarrow$ Accessories $\rightarrow$ Serial port terminal.
Applications $\rightarrow$ Accessories $\rightarrow$ moserial Terminal.
Inside the program, click on the serial port configuration icon.
% Configuration $\rightarrow$ Port.
% See Figure \ref{fig:gtkterm-config}.
See Figure \ref{fig:moserial-config}.

    \begin{figure}[H]
    \centering
        \includegraphics[width=0.8\textwidth]{./install-guide-linux-images/moserial-config.png}
        \caption{Configuring moserial}
        \label{fig:moserial-config}
    \end{figure}


The usual settings are:

    \smallskip
    \begin{tabular}{ l l }
    Port         & /dev/ttyUSB0 \\
    Speed        & 115200       \\
    Parity       & even         \\
    Bits         & 8            \\
    Stop bits    & 1            \\
    Flow control & none         \\
    \end{tabular}
    \smallskip

Note that the port might be different if you are using serial ports on the PC
(in that case, the port could be /dev/ttyS0, /dev/ttyS1, ...)
or if you have more serial devices attached on USB (/dev/ttyUSB1, /dev/ttyUSB2, ...).

After applying appropriate settings, click the first button on the toolbar for
activating the serial port connection.

Now, in Eclipse, select
in the menu
Run $\rightarrow$ External Tools $\rightarrow$ Organize Favorites,
select \emph{Add}, check both items and click \emph{OK} and \emph{OK}.
See Figure \ref{fig:compile-external-tools}.

    \begin{figure}[H]
    \centering
        \includegraphics[width=\textwidth]{./install-guide-linux-images/compile-external-tools.png}
        \caption{Adding external tools}
        \label{fig:compile-external-tools}
    \end{figure}

In order to load the program select
Run $\rightarrow$ External Tools $\rightarrow$ OpenOCD-ARM-USB-OCD Load.

In the console in Eclipse, you should see the messages depicted on
Figure \ref{fig:load-successful},
after about 5 seconds


    \begin{figure}[H]
    \centering
        \includegraphics[width=\textwidth]{./install-guide-linux-images/load-successful.png}
        \caption{Console window contents, after uploading the code to the device}
        \label{fig:load-successful}
    \end{figure}

After these messages appear, VESNA should start printing hello world messages
on the serial terminal, as shown on Figure \ref{fig:hello-world}.

    \begin{figure}[H]
    \centering
        \includegraphics[width=\textwidth]{./install-guide-linux-images/hello-world-cutecom.png}
        \caption{Hello world messages on serial terminal}
        \label{fig:hello-world}
    \end{figure}


By following these steps, you should be able to compile and load program to VESNA.


\subsubsection{Testing debugging facilities}

In order to debug in Eclipse, first the debug configurations have to be added.
In Eclipse, select on the toolbar \emph{Debug} $\rightarrow$ \emph{Organize favorites...}.

In the window that appears, click \emph{Add...},
select both items, and click \emph{OK} and \emph{OK}. Illustration can be found on
\ref{fig:debug-add-fav}.

    \begin{figure}[H]
    \centering
        \includegraphics[width=\textwidth]{./install-guide-linux-images/debug-add-fav.png}
        \caption{Adding debug favorites}
        \label{fig:debug-add-fav}
    \end{figure}

In order to be able to debug programs running on VESNA,
a debug server must be started.
In this case, the debug server is called OpenOCD.
Note that while the debug server is running, VESNA cannot be reprogrammed.
Starting the debug server can be performed by clicking on the
\emph{Debug} icon on the toolbar, and selecting
\emph{OpenOCD-ARM-USB-OCD Debug}.
See Figure \ref{fig:debug-openocd-start}.
Launching the debug server is necessary every time a new debug session is started,
and it must be stopped before reprogramming the VESNA.

    \begin{figure}[H]
    \centering
        \includegraphics[width=\textwidth]{./install-guide-linux-images/debug-openocd-start.png}
        \caption{Starting OpenOCD debug server}
        \label{fig:debug-openocd-start}
    \end{figure}

Next, start the actual program that will be debugged.
For this, click on the \emph{Debug} icon on the toolbar, and select
\emph{OpenOCD-ARM-USB-OCD VESNA Drivers Debug}
(Figure \ref{fig:debug-launch})

    \begin{figure}[H]
    \centering
        \includegraphics[width=\textwidth]{./install-guide-linux-images/debug-launch.png}
        \caption{Launching the program for debugging}
        \label{fig:debug-launch}
    \end{figure}

Now it's recommended to switch to the debug perspective,
for efficient debugging.
Click the \emph{Debug} icon on the upper right corner of the Eclipse window.
The debug perspective is depicted on Figure \ref{fig:debug-window}.


    \begin{figure}[H]
    \centering
        \includegraphics[width=\textwidth]{./install-guide-linux-images/debug-window.png}
        \caption{Debug perspective}
        \label{fig:debug-window}
    \end{figure}

The marked elements on Figure \ref{fig:debug-window} are:
    \begin{itemize}
    \item[(a)] Program flow control and the running processes.
        Note that the list also contains OpenOCD, the debug server,
        and the program being debugged, including the stack(s).
    \item[(b)] Source code view
    \item[(c)] The line being executed.
    \item[(d)] The outline of the current source code.
        In the upper region there are the lists of breakpoints, register values,
        variables.
    \item[(e)] Debugging console.
    \end{itemize}

At the end of debugging, the executed program and also the
debug server should be stopped.

\subsection{Troubleshooting}

This section tries to give fixes to common problems.
The list might be incomplete, so feel free to contact the author for including
new entries in this section.

\subsubsection{Error while starting the debugger: ``Remote 'g' packet is too long: ...''}

\label{sect:gdb-problems}
This problem is caused by incompatible versions of OpenOCD and CodeSourcery debugger.
See a discussion about this topic at OpenOCD mailing list:
\url{https://lists.berlios.de/pipermail/openocd-development/2010-December/017349.html}

Below is a table of OpenOCD and CodeSourcery toolchain versions that are known to work:

    \smallskip
    \begin{tabular}{ l | l }
    CodeSourcery version & OpenOCD version \\
    \hline
    2010q1-188          &   0.4.0-1+nmu1 (Ubuntu 10.10 package) \\
    % add other versions here, if needed
    \end{tabular}
    \smallskip

Table of not working combinations:

    \smallskip
    \begin{tabular}{ l | l }
    CodeSourcery version & OpenOCD version \\
    \hline
    2010.09-51          &   0.4.0-1+nmu1 (Ubuntu 10.10 package) \\
    % add other versions here, if needed
    \end{tabular}
    \smallskip

Combinations not appearing in any of the tables have not been tested.

\subsubsection{OpenOCD prints error about stm32.flash
    when attempting to upload program,
    on Ubuntu 10.04 LTS}

This problem is caused by the old version of OpenOCD (v3.1)
supplied with Ubuntu 10.04.
A simple workaround is to temporarily add the Ubuntu 10.10 package repositories
to the system update sources and install version 4 of OpenOCD from there.
After installing OpenOCD, remove the extra repositories,
because installing software from different version of the same distribution
might cause problems.


\newpage

\section{Command-line way}

The following instructions assume you are working in a terminal emulator,
either a text-mode virtual console or a graphical emulator. If you are using the
default GNOME environment, you can open one from the Applications, Accessories,
Terminal menu entry.

Command lines starting with dollar sign (\verb|$|) are to be executed under your
normal user privileges. Lines starting with a hash sign (\verb|#|) should be
executed as root using \verb|su| or \verb|sudo|, depending on your system
configuration.

A backslash (\verb|\|) at the end of the line means the command continues on the
next line without a carriage return.

\subsection{Installing dependencies}

All of the software that we will compile from source will be installed into
your home directory. This approach was chosen so that the time required to
work with elevated privileges is minimal and hence minimize the chances of the
installation affecting other users or software on your system. However, before
we can do that, we need to install some prerequisite software through the
system's package manager.

Perform the following command as root and confirm installation of packages
when prompted:

\begin{verbatim}
# aptitude install libgmp3-dev \
libncurses5-dev \
libmpc-dev \
autoconf \
build-essential \
git-core
# aptitude build-dep gcc-4.4
\end{verbatim}

\subsection{Installing on-chip debugger}

At the time of writing, the \verb|openocd| version in the Debian stable
distribution (0.3.1) is too old to support VESNA. We require at least version 0.5.0, so
we will compile and install the package from source.

If your distribution already ships with \verb|openocd| version 0.5.0 or later
(for instance Ubuntu 12.10), you can skip this step and simply install
\verb|openocd| through the package manager (e.g. \verb|aptitude install
openocd|).

Begin by downloading and unpacking the Debian source package (if the URLs below
no longer work, you can find a suitable version at
\url{http://packages.debian.org/wheezy/openocd}).

\begin{verbatim}
$ wget http://ftp.de.debian.org/debian/pool/main/o/openocd/openocd_0.5.0-1.dsc \
http://ftp.de.debian.org/debian/pool/main/o/openocd/openocd_0.5.0.orig.tar.bz2 \
http://ftp.de.debian.org/debian/pool/main/o/openocd/openocd_0.5.0-1.debian.tar.gz
$ dpkg-source -x openocd_0.5.0-1.dsc
\end{verbatim}

Now build the package. \verb|dpkg-buildpackage| might display a message about
missing build dependencies. In that case, install the missing packages that are
listed in the error message using \verb|aptitude install| and re-run
\verb|dpkg-buildpackage|.

\begin{verbatim}
$ cd openocd-0.5.0
$ dpkg-buildpackage -uc -us
$ cd ..
\end{verbatim}

Install the compiled package as root:

\begin{verbatim}
# dpkg -i openocd_0.5.0-1_amd64.deb
\end{verbatim}

Check the setup by running \verb|openocd -v|. It should return the following:

\begin{verbatim}
Open On-Chip Debugger 0.5.0 (2011-08-09-08:45)
Licensed under GNU GPL v2
For bug reports, read
	http://openocd.berlios.de/doc/doxygen/bugs.html
\end{verbatim}

Note that your user account usually needs to be a member of the group
\verb|plugdev| in order to have access to the OCD programmer and other USB
devices. You can ensure this by running the following command as root (replace
\verb|foo| with your user name):

\begin{verbatim}
# adduser foo plugdev
\end{verbatim}

\subsection{Installing cross-compiler toolchain}

We will use a customized version of the \verb|summon-arm-toolchain| script to
build a GNU toolchain targeting ARM Cortex M3 architecture.

First we need to download the latest version of the script:

\begin{verbatim}
$ git clone https://github.com/avian2/summon-arm-toolchain.git
\end{verbatim}

Now run the script. This will download, compile and install the toolchain
(binutils, gcc, gdb and newlib C library) into \verb|~/local|. If you
would like to use a different directory, pass a different path to the
\verb|-t| command-line argument.

\begin{verbatim}
$ cd summon-arm-toolchain
$ ./summon-arm-toolchain -t ~/local
\end{verbatim}

You should now add \verb|~/local/bin| (or the directory you installed the toolchain
in) to your PATH. This is best done by editing \verb|~/.bashrc| and adding a
line like the following at the top:

\begin{verbatim}
export PATH="$HOME/local/bin:$PATH"
\end{verbatim}

Remember that this setting will take effect only after you open a new terminal window or
a new login.

To test the setup, try running \verb|arm-none-eabi-gcc --version|. It should have the
following effect:

\begin{verbatim}
$ arm-none-eabi-gcc --version
arm-none-eabi-gcc (GCC) 4.6.3
Copyright (C) 2011 Free Software Foundation, Inc.
This is free software; see the source for copying conditions.  There is NO
warranty; not even for MERCHANTABILITY or FITNESS FOR A PARTICULAR PURPOSE.
\end{verbatim}

\subsection{Compiling libopencm3}

Libopencm3 is a standard peripherals support library for ARM Cortex M3
microcontrollers. The version used below contains patches for VESNA that have
not yet been accepted up-stream.

\begin{verbatim}
$ git clone https://github.com/avian2/libopencm3.git
\end{verbatim}

Now compile and install the library. Adjust the DESTDIR value to the same prefix
you used when installing the toolchain.

\begin{verbatim}
$ cd libopencm3
$ make
$ DESTDIR=~/local make install
\end{verbatim}

\subsection{Testing the setup}

To test the setup we will compile, upload and debug a short program that blinks
the yellow LED present on the VESNA core board and sends a message to debugger's
serial port.

First, download the source of the program:

\begin{verbatim}
$ git clone https://github.com/avian2/vesna-hello-world.git
\end{verbatim}

Then compile it:

\begin{verbatim}
$ cd vesna-hello-world
$ make
\end{verbatim}

And load it into microcontroller's flash. For this to work, you need to have an
Olimex ARM-USB-OCD programmer connected to VESNA through the SND debug board.
Make sure the power and JTAG connectors are securely attached.

\begin{verbatim}
$ make hello-world.u
\end{verbatim}

The yellow LED on the VESNA board should start blinking after programming is complete.
To see messages sent to the serial port, you need to attach the serial
connector on the Olimex ARM-USB-OCD to the VESNA's serial port.

You then need to read the correct device file. If you don't have any
other USB-to-serial adapters connected to the machine, this will usually be
\verb|/dev/ttyUSB0|. The program sets the baud rate to 19200 bps, and we have to
match that setting on the Linux side.

\begin{verbatim}
$ stty -F /dev/ttyUSB0 19200
$ cat /dev/ttyUSB0
\end{verbatim}

You should now see a stream of \verb|Hello, world!| messages appear in the
terminal. Of course, you can also use your serial terminal emulator of choice

Note that on most setups your user account needs to be in the
\verb|dialout| group in order to have access to serial adapters.

\subsection{Debugging setup}

OpenOCD allows you to attach a GNU debugger instance to VESNA. This allows you
to conveniently debug the code running on the microcontroller, for instance by
manually inspecting memory contents or single stepping through code.

We begin by starting OpenOCD in server mode.

\begin{verbatim}
$ openocd -f interface/olimex-arm-usb-ocd.cfg -f target/stm32f1x.cfg
\end{verbatim}

Then we can start GNU debugger and attach it to OpenOCD.

\begin{verbatim}
$ arm-none-eabi-gdb
(gdb) target remote localhost:3333
\end{verbatim}

At this point we can start debugging the code that is already running on the
microcontroller using the usual GDB commands. To reset the microcontroller, use:

\begin{verbatim}
(gdb) monitor reset halt
(gdb) continue
\end{verbatim}

You can also program the microcontroller's flash directly from the debugger. For
instance, to load the hello-world program, do:

\begin{verbatim}
(gdb) monitor reset halt
(gdb) file hello-world.elf
(gdb) load
(gdb) monitor reset halt
\end{verbatim}

Note that to see the source of the libopencm3 functions correctly, you might
need to adjust the source search path. Adjust the path in the command below to
point to the location where you compiled libopencm3.

\begin{verbatim}
(gdb) directory ~/libopencm3/lib/stm32/f1
\end{verbatim}

\newpage

\subsection{Troubleshooting}

This section tries to give solutions to common problems.  The list might be
incomplete, so feel free to contact the author for including new entries in
this section.

\subsubsection{unable to open ftdi device: device not found}

\verb|openocd| will report this error if initialization failed, even if
the USB device itself was actually found. Common reasons are access denied
errors due to bad device file permissions or ioctl errors because OpenOCD
expects a different type of hardware.

All instructions above assume you are using the Olimex ARM-USB-OCD
programmer. Note that Olimex ARM-USB-OCD-H is a different device, requiring
a different OpenOCD configuration.

If you are using a different programmer, you have to set the
\verb|OPENOCD_PARAMS| environment accordingly. For instance, for Olimex
ARM-USB-OCD-H, the following line can be added to the top of the
\verb|.bashrc| file:

\begin{verbatim}
export OPENOCD_PARAMS="-f interface/olimex-arm-usb-ocd-h.cfg \
-f target/stm32f1x.cfg"
\end{verbatim}

Also note that at the time of writing, \verb|openocd| packages for Debian
and Ubuntu do not include proper \verb|udev| rules for ARM-USB-OCD-H
hardware and hence do not set correct permissions for device files in that
case. This can be corrected by adding the following configuration to a file
named \\ \verb|/etc/udev/rules.d/40-openocd.rules| (unplug the
programmer beforehand):

\begin{verbatim}
ACTION!="add|change", GOTO="openocd_rules_end"
SUBSYSTEM!="usb", GOTO="openocd_rules_end"
ENV{DEVTYPE}!="usb_device", GOTO="openocd_rules_end"

# Olimex ARM-USB-OCD-H
ATTRS{idVendor}=="15ba", ATTRS{idProduct}=="002b", MODE="664", \
GROUP="plugdev"

LABEL="openocd_rules_end"
\end{verbatim}

\begin{thebibliography}{9}
\bibitem{mm10}
	% FIXME update reference manual to "VESNA" manual
   Marko Mihelin, SensorLab VESNA open source development environment setup manual for Windows based development,
   2011

\end{thebibliography}

\end{document}
